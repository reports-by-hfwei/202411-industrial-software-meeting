% db-history.tex

%%%%%%%%%%%%%%%%%%%%
\begin{frame}{}
  \begin{itemize}
    \centering
    \item \blue{\bf 数据库系统}实现正确了吗?
  \end{itemize}

  \begin{center}
    \fig{width = 0.50\textwidth}{figs/correctness}
  \end{center}

	\begin{center}
		数据库系统通过\cyan{事务}与\purple{\bf 隔离级别} (Isolation Levels) 保障\red{数据的正确性}
	\end{center}
\end{frame}
%%%%%%%%%%%%%%%%%%%%

%%%%%%%%%%%%%%%%%%%%
\begin{frame}{}
	\begin{center}
		声称实现了\blue{\bf 快照隔离} (Snapshot Isolation; SI)

		\fig{width = 0.80\textwidth}{figs/db-si}
	\end{center}
\end{frame}
%%%%%%%%%%%%%%%%%%%%

%%%%%%%%%%%%%%%%%%%%
\begin{frame}{}
	\begin{center}
		但是, 仍存在违反\blue{\bf 快照隔离}的\cyan{\bf 数据异常} (Data Anomalies) 情况

		\fig{width = 0.95\textwidth}{figs/db-si-violations}
	\end{center}
\end{frame}
%%%%%%%%%%%%%%%%%%%%

%%%%%%%%%%%%%%%%%%%%
\begin{frame}{}
	\begin{center}
		{\Large 黑盒测试: 数据库系统\red{\bf 执行历史}验证问题}

		\fig{width = 0.90\textwidth}{figs/problem}
	\end{center}
\end{frame}
%%%%%%%%%%%%%%%%%%%%

%%%%%%%%%%%%%%%%%%%%
\begin{frame}{}
	\begin{center}
		\fig{width = 0.90\textwidth}{figs/polysi-checker-conclusion}

		\vspace{5pt}
		\ncite{PolySI@VLDB'2023}

		\fig{width = 0.20\textwidth}{figs/sat-solver}
	\end{center}
\end{frame}
%%%%%%%%%%%%%%%%%%%%

%%%%%%%%%%%%%%%%%%%%
\begin{frame}{}
	\begin{columns}
		\column{0.33\textwidth}
		  \fig{width = 1.00\textwidth}{figs/polysi-runtime}
			\begin{center}
				时间
			\end{center}
		\column{0.33\textwidth}
		  \fig{width = 1.00\textwidth}{figs/polysi-memory}
			\begin{center}
				内存
			\end{center}
		\column{0.33\textwidth}
		  \fig{width = 1.00\textwidth}{figs/polysi-scalability}
			\begin{center}
				可扩展性
			\end{center}
	\end{columns}
\end{frame}
%%%%%%%%%%%%%%%%%%%%

%%%%%%%%%%%%%%%%%%%%
\begin{frame}{}
	\begin{center}
		\fig{width = 0.90\textwidth}{figs/hierarchy}

		\vspace{5pt}
		\ncite{Plume@OOPSLA'2024}
	\end{center}
\end{frame}
%%%%%%%%%%%%%%%%%%%%

%%%%%%%%%%%%%%%%%%%%
% \begin{frame}{}
% 	\begin{center}
% 		\fig{width = 1.00\textwidth}{figs/taps}
% 	\end{center}
% \end{frame}
%%%%%%%%%%%%%%%%%%%%

%%%%%%%%%%%%%%%%%%%%
% \begin{frame}{}
% 	\begin{columns}
% 		\column{0.50\textwidth}
% 			\fig{width = 1.00\textwidth}{figs/taps}
% 		\column{0.50\textwidth}
% 			\fig{width = 1.00\textwidth}{figs/taps-vis}
% 	\end{columns}
% \end{frame}
%%%%%%%%%%%%%%%%%%%%

%%%%%%%%%%%%%%%%%%%%
\begin{frame}{}
	\begin{center}
		如何利用\blue{\bf 数据异常}\red{\bf 全面、准确}地定义这些事务隔离级别
	\end{center}

	\begin{center}
		\fig{width = 0.40\textwidth}{figs/iso-taps}
	\end{center}

	\begin{columns}
		\column{0.50\textwidth}
			\fig{width = 1.00\textwidth}{figs/taps}
		\column{0.50\textwidth}
			\fig{width = 1.00\textwidth}{figs/taps-vis}
	\end{columns}
\end{frame}
%%%%%%%%%%%%%%%%%%%%

%%%%%%%%%%%%%%%%%%%%
\begin{frame}{}
	\fig{width = 0.90\textwidth}{figs/taps-exps}
\end{frame}
%%%%%%%%%%%%%%%%%%%%

%%%%%%%%%%%%%%%%%%%%
\begin{frame}{}
	\fig{width = 0.90\textwidth}{figs/taps-bugs}
\end{frame}
%%%%%%%%%%%%%%%%%%%%

%%%%%%%%%%%%%%%%%%%%
\begin{frame}{}
	\fig{width = 0.80\textwidth}{figs/plume-time}
	\fig{width = 0.80\textwidth}{figs/plume-memory}
\end{frame}
%%%%%%%%%%%%%%%%%%%%

%%%%%%%%%%%%%%%%%%%%
\begin{frame}{}
	\begin{center}
		\fig{width = 0.70\textwidth}{figs/isovista}

		\vspace{5pt}
		\ncite{IsoVista@VLDB'2024 (Demo)}
	\end{center}
\end{frame}
%%%%%%%%%%%%%%%%%%%%